\chapter{Metodología}

Para lograr los objetivos propuestos se utilizará la metodología de desarrollo en cascada \cite{Moha}, en las 4 fases descritas a continuación:

\section{Fase 1. Recolección de información} 
	Para la etapa inicial de este proyecto se consultaron diversas fuentes bibliográficas relacionadas con IoT y Smart House, tales como artículos científicos, libros, revistas, sitios de internet e incluso cursos en línea como “Internet del todo” de Cisco NetAcad. Por otra parte, se realizó una investigación sobre diferentes sistemas embebidos y módulos disponibles en el mercado, así como sus especificaciones técnicas, permitiendo una visión más amplia de las posibilidades. \\

\section{Fase 2. Desarrollo de hardware}
	Para el desarrollo del hardware, se tomará como base principal un sistema embebido con diferentes funcionalidades, idóneas para implementar en IoT, entre las cuales estará la capacidad de conexión inalámbrica (WiFi) , posteriormente se identifican los puertos y su tipo para los cumplir con los requerimientos de la habitación, esto es, teniendo en cuenta las posibles cargas a implementar, ya sean AC o DC. \\

\section{Fase 3. Desarrollo de software}
	Para esta etapa, se identifican diferentes servidores que permitan dar soporte al desarrollo del software, es decir, una aplicación web. A continuación, se procede a construir la aplicación de monitoreo y control para el entorno, la cual estará vinculada a el dispositivo IoT desarrollado.\\

\section{Fase 4. Presentación de resultados} 
	En la fase final del proyecto se instala el hardware y el software en una habitación real, o en un prototipo de habitación para realizar una evaluación del sistema en una prueba beta, con el fin de mostrar los resultados obtenidos y verificar el cumplimiento de los requerimientos.\\