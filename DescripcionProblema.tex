\chapter{Descripción del Problema}

Se ha mencionado en este documento, que las investigaciones sobre Smart House están dirigidas a sistemas comúnmente compuestos de 3 partes, las cuales son, hardware, software y firmware. En cuanto a hardware, la mayor parte de de estas investigaciones se ven limitadas a la implementación de tarjetas de prototipado como unidad de control, llegando a subutilizarlas, esto es, debido a que es un dispositivo genérico diseñado para tareas generales, por esto, puede contener periféricos sin utilidad para ciertas tareas específicas de un sistema, así como puede carecer de otros que son requeridos.\\

Para el entorno de Smart House, es necesario contar con conectividad inalámbrica, puesto que la interconexión con múltiples dispositivos es un objetivo de estos entornos. A pesar de esto, la mayoría de estas tarjetas carece de esta conectividad, lo que hace necesario adquirir o implementar módulos que la vinculen a la red. Asi como tambien, las tarjetas que se usan en este entorno y que se encuentran actualmente en el mercado, no contienen etapas de potencia. Sin embargo, es posible encontrar hardware en módulos con estas etapas de potencia, que se ajusta a las tarjetas, para manejar las distintas cargas que se encuentran en la casa, conectadas directamente a la red eléctrica, como por ejemplo, ventiladores, bombillos, entre otras.\\

Otro inconveniente que se puede presentar en estas tarjetas, es la capacidad de procesamiento, la cual puede estar en las mismas condiciones, es decir, muy poco procesamiento o por el contrario, demasiado procesamiento para tareas simples. Este factor está relacionado directamente a los costos, es decir, las características generales de este tipo de tarjetas, las hacen tener herramientas que pueden no ser requeridas en múltiples tareas, traduciendo este desuso en dinero invertido sin beneficios.\\

Con la falta de un consenso ante el campo de IoT, existe dificultad para implementar nuevas soluciones a los problemas que este campo intenta resolver, gran parte de ellos se debe a los costos elevados de los múltiples dispositivos en el mercado. De esta manera, desarrollar un sistema IoT para una habitación el cual se pueda ser escalable para entornos amplios, que requieren gran cantidad de estos dispositivos, como empresas o centros comerciales, se convierte en un desafío monetario, de escalabilidad y de compatibilidad. Todas estas dificultades se convierten en barreras para la mayor parte de los hogares en países en desarrollo como en Colombia.\\

En este orden de ideas, el desafío para desarrollar un entorno Smart House radica en la escasez de sistemas específicos, con capacidades enfocadas a las tareas fundamentales que requiere este tipo de soluciones en una habitación vista como la mínima unidad de la casa, así como también la baja accesibilidad que ofrece el mercado a algunas de las soluciones existentes. En consecuencia, los dispositivos de Smart House disponibles en el mercado, no se enfocan en la escalabilidad en cuanto a los costos de implementación y algunos dispositvos tampoco se enfocan en la compatibilidad, dejando de lado estos conceptos, que son muy importantes en relación a la instalación de los múltiples dispositivos a usar, ya que esta aplicación es específica de IoT. Teniendo en cuenta que una de las estimaciones para la tecnología IoT, está orientada hacia un mundo con millones de objetos conectados, como los controladores, sensores, actuadores y demás.\\
